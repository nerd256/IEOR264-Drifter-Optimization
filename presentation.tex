\documentclass{beamer}
\usetheme{default}

\setbeamercolor{umbcboxes}{bg=violet!12,fg=black}

\usepackage{rotating} % for defining \schwa
\newcommand{\schwa}{\raisebox{1ex}{\begin{turn}{180}e\end{turn}}}

\newcommand{\arcsinh}{\mathop\mathrm{arcsinh}\nolimits}
\newcommand{\arccosh}{\mathop\mathrm{arccosh}\nolimits}
\newcommand{\Pu}{P_{\mathrm{amb}}}

\title{Path-Selection in Floating Sensor Networks}
\author[K. Weekly]{Kevin Weekly}
\institute[UCB]{
  Dept. of Electrical Engineering and Computer Sciences\\
  University of California, Berkeley \\
}
\date{April 15, 2011}

\begin{document}
\begin{frame}[plain]
  \titlepage
\end{frame}


\begin{frame}{Outline}

\begin{enumerate}
  \item Problem setup
 \begin{enumerate}
  \item Link Labelling
  \item Path enumeration
  \item Path/Region costs
 \end{enumerate}

  \item \{0,1\} Linear Programs
 \begin{enumerate}
  \item All links minimum units
  \item Max links fixed units
  \item All links/All time minumum units
  \item Max links/All time
 \end{enumerate}

  \item Mixed Integer Quadratic Programs
  \begin{enumerate}
  \item Increasing marginal utility
  \item Decreasing marginal utility
  \end{enumerate}
\end{enumerate}
\end{frame}

\begin{frame}
\begin{theorem}
asdf
\end{theorem}
\end{frame}


\end{document}