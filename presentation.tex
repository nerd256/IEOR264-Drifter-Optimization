\documentclass[xcolor=pdflatex,dvipsnames,table]{beamer}
\usepackage{algorithmic}
\usepackage{multimedia}

\usetheme{default}



\setbeamercolor{umbcboxes}{bg=violet!12,fg=black}

\usepackage{rotating} % for defining \schwa
\newcommand{\schwa}{\raisebox{1ex}{\begin{turn}{180}e\end{turn}}}

\newcommand{\arcsinh}{\mathop\mathrm{arcsinh}\nolimits}
\newcommand{\arccosh}{\mathop\mathrm{arccosh}\nolimits}
\newcommand{\Pu}{P_{\mathrm{amb}}}

\title{Path-Selection in Floating Sensor Networks}
\author[K. Weekly]{Kevin Weekly}
\institute[UCB]{
  Dept. of Electrical Engineering and Computer Sciences\\
  University of California, Berkeley \\
}
\date{April 15, 2011}

\begin{document}

\begin{frame}[plain]
  \titlepage
\end{frame}

%%%%%%%%%%%%%%%%%%%%% INTRODUCTION %%%%%%%%%%%%%%%%%%%%%%%%%%%%%%%%%%%%%%%%%%%%%%%%%%%%%%%
\section{Introduction}
\begin{frame}{Introduction to the Floating Sensor Network}
 

\end{frame}


%% We want first outline to come after introduction slide
\AtBeginSection[]
{
   \begin{frame}
       \tableofcontents[currentsection]
   \end{frame}
}

%%%%%%%%%%%%%%%%%%%%% LINK LABELLING %%%%%%%%%%%%%%%%%%%%%%%%%%%%%%%%%%%%%%%%%%%%%%%%%%%%%%%
\section{Problem Setup}
\subsection{Link Labelling}
\begin{frame}{Problem Setup: Link Labelling}
Breadth-First-Search (BFS) style flood-fill:
\begin{itemize}
   \item Maintains edge list of water pixels to process and processes them in BFS order.
   \item Each edge is given a label. As it sweeps through the domain, that label is assigned to underlying pixels.
   \item Monitors continuity of edges.
\begin{itemize}
    \item Path Splitting: If a previously continuous edge becomes broken, two new labels are generated and connections are recorded. 
    \item Path Joining: If two edges meet, a new label is generated and connections are recorded.
\end{itemize}
   \item Algorithm ends when there are no more water pixels to process.
\end{itemize}
\end{frame}

\begin{frame}{Problem Setup: Link Labelling}
\begin{columns}[c]
\column{0.5\linewidth}
San Joaquin/Sacramento River Delta Target Domain
\begin{itemize}
 \item Approx. 2km$\times$8km 
 \item 105 labelled regions after pruning 0-area regions
\end{itemize}

Weaknesses:
\begin{itemize}
 \item ``King's move'' dynamics.
 \item Bias towards coordinate system.
 \item Several long strips.
 \item Joining of labels relies on timing.
\end{itemize}


\column{0.5\linewidth}
  \begin{figure}
  \includegraphics[width=1\linewidth]{figures/domain.png}\\
  \movie[width=0.5\linewidth, height=0.5\textheight, rate=2, autostart, autoplay, repeat, poster]{}{figures/anim.flv}
  \hfill
  \includegraphics[height=0.5\textheight]{figures/labelled.png}
  \end{figure}
\end{columns}
\end{frame}


%%%%%%%%%%%%%%%%%%%%% PATH ENUMERATION %%%%%%%%%%%%%%%%%%%%%%%%%%%%%%%%%%%%%%%%%%%%%%%%%%%%%%%
\subsection{Path Enumeration}
\begin{frame}{Problem Setup: Path Enumeration}
\begin{columns}
 \small
 \column{0.5\linewidth}
Depth-First-Search (DFS) through connection list:
\begin{itemize}
 \item Exhaustively enumerates all possible paths to predetermined ``sink'' labels.
 \item Connections are directed based on order they were encountered during labelling-- a path cannot backtrack.
 \item If a dead-end is reached, DFS attempts to ``punch-through'' a connection directed against it (that was not already visited).
\end{itemize}

 \column{0.5\linewidth}
\begin{tabular}{c|l}
 \includegraphics[height=0.7\textheight]{figures/labelled.png} &
 \small {1547 paths found}
\end{tabular}

\end{columns}

\end{frame}

%%%%%%%%%%%%%%%%%%%%% STATIC FORMULATION %%%%%%%%%%%%%%%%%%%%%%%%%%%%%%%%%%%%%%%%%%%%%%%%%%%%%%%
\section{\{0,1\} Static Linear Programs}
\subsection{Formulation}
\begin{frame}{\{0,1\} Static Linear Programs: Formulation}
\begin{align*}
\max_{x,y}\;& -g^Tx + h^Ty \\
\mbox{s.t.}\;& \left[ \begin{array}{c}
                       A\\
                       -1, \dots, -1
                      \end{array}\right] x \geq \left[ \begin{array}{c} y\\-D \end{array}\right] \\
&a_{ij} = \begin{cases}
           1\;\mbox{if path $i$ contains link $j$}\\
           0\;\mbox{otherwise}
          \end{cases}\\
&x_i \in \{0,1\}\;\forall i\in\{1\dots n\}\\
&y_j \in \{0,1\}\;\forall j\in\{1\dots m\}
\end{align*}

\begin{itemize}
\item $\{x_i\}$ : indicates whether path $i$ of $n$ total paths is taken.
\item $\{y_j\}$ : indicates whether link $j$ of $m$ total links is visited.
\item $D$ : total number of units avaiable.
\item Preprocessing step : Remove all links impossible to visit (zero-rows of A).
\end{itemize}

\end{frame}


%%%%%%%%%%%%%%%%%%%%%% ALL LINKS MINIMUM UNITS %%%%%%%%%%%%%%%%%%%%%%%%%%%%%%%%%%%%%%%%%%%%%%
\subsection{All links minimum units}
\begin{frame}{\{0,1\} Static Linear Programs: All links minimum units}
\emph{``Visit every link using the least number of drifters''}
\begin{align*}
g &= \left[ 1,1, \dots ,1\right]\\
y &= \left[ 1,1, \dots ,1\right]\\
D &= \infty \;\mbox{(remove row from ILP)}
\end{align*}

\begin{itemize}
 \item Preprocessing : Find paths which must be taken ( rows of A containing only one 1 ). Remove these paths and all of their links from ILP. Repeat until all rows of A have two or more 1s.
\end{itemize}
\end{frame}

\begin{frame}{\{0,1\} Static Linear Programs: All links minimum units}
\begin{figure}
\rowcolors{1}{RoyalBlue!20}{RoyalBlue!5}
  \begin{tabular}{|l|c|}
  \hline 
  Paths in ILP/Total & 1544/1547 \\
  Labels in ILP/Total & 70/105 \\
  CPU Time (cplex) & 250mS\\
  CPU Time (glpk) & 180mS \\
  Units Needed & 10 \\
  \hline
  \end{tabular}
\end{figure}

  \begin{figure}
     {\includegraphics[width=1\textwidth]{figures/all_links_minimum_units_paths.png}}
  \end{figure}
\end{frame}


%%%%%%%%%%%%%%%%%%%%%% MAX COVERAGE FIXED UNITS %%%%%%%%%%%%%%%%%%%%%%%%%%%%%%%%%%%%%%%%%%%%%
\subsection{Max coverage fixed units}
\begin{frame}{\{0,1\} Static Linear Programs: Max coverage fixed units}
\emph{``Visit the greatest total weighted area with a fixed number of units.''}
\begin{align*}
g &= 0\\
h &= \left[ w_1,w_2, \dots ,w_m\right]\\
D &= 5\\
\end{align*}

\begin{itemize}
 \item Path processing : Set $w_j$ to the weight of each region with label $j$.
 \item Note: Setting $w_j=1$ solves the problem\\ \emph{``Visit the most number of links with a fixed number of units.''}
\end{itemize}
\end{frame}

\begin{frame}{\{0,1\} Static Linear Programs: Max coverage fixed units}
Example: $w_j$ are the number of pixels in each region.
\begin{figure}
\rowcolors{1}{RoyalBlue!20}{RoyalBlue!5}
  \begin{tabular}{|l|c|}
  \hline 
  Paths & 1547 \\
  Labels in ILP/Total & 82/105 \\
  CPU Time (cplex) & 440mS\\
  CPU Time (glpk) & 210mS \\
  Allotted Units & 5\\
  Pixels covered/Total & 31169/32033 (97\%)\\
  \hline
  \end{tabular}
\end{figure}

  \begin{figure}
     {\includegraphics[width=0.5\textwidth]{figures/max_coverage_fixed_units_paths.png}}
  \end{figure}
\end{frame}

%%%%%%%%%%%%%%%%%%%%% DYNAMIC FORMULATION %%%%%%%%%%%%%%%%%%%%%%%%%%%%%%%%%%%%%%%%%%%%%%%%%%%%%%%

\section{\{0,1\} Dynamic Linear Programs}
\subsection{Formulation}
\begin{frame}{\{0,1\} Dynamic Linear Programs: Formulation}
\begin{align*}
\max_{x,y}\;& -g^T \tilde{x} + h^T \tilde{y} \\
\mbox{s.t.}\;& \left[ \begin{array}{c}
                       \tilde{A}\\
                       -1, \dots, -1
                      \end{array}\right] \tilde{x} \geq \left[ \begin{array}{c} \tilde{y}\\-D \end{array}\right] \\
&a_{i,\mbox{\tiny map}(j,t)} = \begin{cases}
           1\;\mbox{if path $i$ passes link $j$ at time $t$}\\
           0\;\mbox{otherwise}
          \end{cases}\\
&\tilde{x}_i \in \{0,1\}\;\forall i\in\{1\dots \tilde{n}\}\\
&\tilde{y}_{\mbox{\tiny map}(j,t)} \in \{0,1\}\;\forall j\in\{1\dots m\},t\in\{1\dots T\}
\end{align*}

\begin{enumerate}
\small
\item Set the time-horizon, $T$, to the maximum path length.
\item Enumerate all paths which can delay at the source node and still arrive at a sink node by $T$. 
\item $\tilde{x}$ now contains original $x$ plus all ``delayed'' paths.
\item $\tilde{y}$ indicates which links are visited and \emph{when}.
\item Preprocessing step : Remove all zero-rows of A.
\end{enumerate}
\end{frame}

%%%%%%%%%%%%%%%%%%%%% ALL TIME ALL PATHS MINIMUM UNITS %%%%%%%%%%%%%%%%%%%%%%%%%%%%%%%%%%%%%%%%%%%%%%%%%%%%%%%
\subsection{All time all links minimum units}
\begin{frame}{\{0,1\} Dynamic Linear Programs: All time all links minimum units}
\begin{columns}

\column{0.5\linewidth}
  \emph{``Within the time it takes to traverse the entire domain, maintain observation of as many regions as feasibly possible, using the minimum number of units. Units may only delay at the drop-off or pick-up points.''}
  \begin{figure}
  \rowcolors{1}{RoyalBlue!20}{RoyalBlue!5}
    \begin{tabular}{|l|c|}
    \hline 
    Paths in ILP/Total & 17328/17487 \\
    Labels in ILP/Total & 1308/4935 \\
    CPU Time (problem setup) & 50s \\
    CPU Time (cplex) & 5s \\
    CPU Time (glpk) & 128s \\
    Time Horizon & 47 steps\\
    Units Needed & 311 \\
    \hline
    \end{tabular}
  \end{figure}


\column{0.5\linewidth}
  \begin{figure}
    \centering
    \hfill
    \movie[width=0.5\linewidth, height=0.7\textheight, rate=2, autostart, autoplay, repeat, poster]{}{figures/all_time_all_links_min_units.flv}
  \end{figure}

\end{columns}
\end{frame}

%%%%%%%%%%%%%%%%%%%%% ALL TIME MAX COVERAGE FIXED UNITS %%%%%%%%%%%%%%%%%%%%%%%%%%%%%%%%%%%%%%%%%%%%%%%%%%%%%%%
\subsection{All time all links minimum units}
\begin{frame}{\{0,1\} Dynamic Linear Programs: All time max coverage fixed units}
\begin{columns}

\column{0.5\linewidth}
  \emph{``Over the time it takes to traverse the entire domain, maximize the total area observed given a fixed number of units. Units may only delay at the drop-off or pick-up points.''}
  \begin{figure}
  \rowcolors{1}{RoyalBlue!20}{RoyalBlue!5}
    \begin{tabular}{|l|c|}
    \hline 
    Paths in ILP & 17487 \\
    Labels in ILP/Total & 2028/4935 \\
    CPU Time (problem setup) & 51s \\
    CPU Time (cplex) & 8s \\
    CPU Time (glpk) & 6s \\
    Time Horizon & 47 steps\\
    Units Allotted & 40\\
    Pixels Covered/Total & 71k/99k (72\%)\\
    \hline
    \end{tabular}
  \end{figure}


\column{0.5\linewidth}
  \begin{figure}
    \centering
    \hfill
    \movie[width=0.5\linewidth, height=0.7\textheight, rate=2, autostart, autoplay, repeat, poster]{}{figures/all_time_max_coverage_fixed_units.flv}
  \end{figure}
\end{columns}
\end{frame}


%%%%%%%%%%%%%%%%%%%%%%% CONCLUSION %%%%%%%%%%%%%%%%%%%%%%%%%%%%%%%%%%%%%%%%%%%%%%%%%%%%%%%%%%%%%%%%%%%%%%%%
\section{Conclusion}
\begin{frame}{Conclusion}
 
\end{frame}



\end{document}